% !TeX spellcheck = es_ES
% Chapter 1

%\chapter{Chapter Title Here} % Main chapter title
%
%\label{Chapter1} % For referencing the chapter elsewhere, use \ref{Chapter1} 

%----------------------------------------------------------------------------------------

% Define some commands to keep the formatting separated from the content 
%\newcommand{\keyword}[1]{\textit{#1}}
%\newcommand{\tabhead}[1]{\textbf{#1}}
%\newcommand{\code}[1]{\texttt{#1}}
%\newcommand{\file}[1]{\texttt{\bfseries#1}}
%\newcommand{\option}[1]{\texttt{\itshape#1}}

%----------------------------------------------------------------------------------------

\chapter{Conclusiones}
Actualmente el calorímetro 2277 Thermal Activity Monitor se encuentra en funcionamiento y en capacidad de
realizar mediciones en un amplio rango de investigaciones en el ámbito de la termodinámica de soluciones, la bioqu\'imica y la qu\'imica en general. Se realiz\'o la instalaci\'on de una celda de medici\'on al cuerpo principal del calor\'imetro, adem\'as de adaptar el equipo a la red eléctrica colombiana con el fin de lograr
hacer uso del mismo en la Universidad de Los Andes. Además, se logró la comunicaci\'on del calor\'imetro con el computador de laboratorio eliminando la necesidad de puertos RS232. El funcionamiento del agitador fue optimizado al sustituir la conexión original a un puerto USB, haciendo de esta una conexión mas práctica, adem\'as, se caracteriz\'o el efecto del mismo sobre las lecturas calorim\'etricas. 

Adicionalmente, y a manera de protocolo se muestran los pasos a seguir para la visualización de los datos en tiempo real, la configuración del calor\'imetro, su calibración y posterior utilización aplicada a un método experimental, permitiendo de esta manera tener en cuenta las consideraciones necesarias para la utilización del calorímetro en métodos experimentales. Por otro lado, cabe resaltar que la incorporación de un sistema de monitoreo por medio de un circuito capaz de integrar las señales de temperatura externa, interna y ambiental, constituye un factor importante en la optimización de las temperaturas de trabajo del calor\'imetro. 

Por otra parte, el trabajo aquí presentado proporciona valores de referencia para la estabilización de la temperatura a 25 \grad{} dadas las condiciones climáticas de Bogotá, lo cuál permite a su vez, reducir el tiempo de experimentación con el mismo. Además, las calibraciones eléctricas del calorímetro permiten reducir los errores sistem\'aticos y que la información proporcionada por este sea consistente con el valor de potencia trabajado para el canal de medición específico.

%A\'un en el caso de inyecciones \'unicas, cuando los resultados est\'an considerablemente alejados de los esperados, los valores obtenidos para titulaciones con el calorímetro se encuentran dentro de los valores reportador en la literatura. Adem\'as, se considera que las posibles discrepancias est\'an asociadas al sistema de inyecci\'on y no al sistema de cuantificaci\'on.