% !TeX spellcheck = es_ANY
% Chapter 1

%\chapter{Chapter Title Here} % Main chapter title
%
%\label{Chapter1} % For referencing the chapter elsewhere, use \ref{Chapter1} 

%----------------------------------------------------------------------------------------

% Define some commands to keep the formatting separated from the content 
%\newcommand{\keyword}[1]{\textit{#1}}
%\newcommand{\tabhead}[1]{\textbf{#1}}
%\newcommand{\code}[1]{\texttt{#1}}
%\newcommand{\file}[1]{\texttt{\bfseries#1}}
%\newcommand{\option}[1]{\texttt{\itshape#1}}

%----------------------------------------------------------------------------------------

\chapter{Calibraci\'on Qu\'imica}
	La calibraci\'on qu\'imica consiste en contrastar las propiedades termodin\'amicas de un sistema qu\'imico, obtenidas usando el calor\'imetro con aquellas reportadas en la literatura. Para esto se estudia la reacci\'on del \'acido clorhidrico con bicarbonato de potasio. Esta reacci\'on presenta varias ventajas: por un lado hace uso de reactivos de f\'acil acceso, es una reacci\'on con esqueometr\'ia 1:1, ha sido estudiada previamente, y adem\'as, es usada en calibraciones de equipos calorim\'etricos., como el calor\'imetro de titulaci\'on NanoITC de \textit{TA instruments}.
	
\section{Preparaci\'on de las soluciones de HCl}\label{sec: soluciones}
	Para la preparaci\'on de las soluciones, se hace uso de forma sistem\'atica de una balanza XXXX y un dens\'imetro YYYY. Adem\'as, en el proceso de preparaci\'on se hace necesaria la toma de dos al\'icuotas de 30 $\mu$L y 170 $\mu$L, para el HCl y \ce{KHCO3} correspondientemente, por lo cual se usa una micropipeta ZZZZ con rango de 20 a 200 $\mu$L.
	\subsection{Soluci\'on de HCl 0.25 mM}
		Para determinar la concentraci\'on de una soluci\'on de 1,0 mL de HCl concentrado en 25,0 mL de agua tipo 1, se midi\'o la densidad de la soluci\'on, posteriormente usando como referencia los datos reportados en la literatura fue realizada una regresi\'on lineal que permiti\'o relacionar la concentraci\'on con la densidad de la soluci\'on $\rho$ \cite{perry2007perry}. De esta manera se estableci\'o el valor de la concentraci\'on en: $3.02 \pm 0.05$ \% (fracci\'on de masa $[w_t]$), donde la incertidumbre se obtiene de la pendiente ($m$) e intercepto ($b$) de la regresi\'on lineal que se muestra en la \autoref{fig: HCl_density}.
		\begin{equation}
			\delta [w_t] = \sqrt{\left(\dfrac{\rho-b}{m^2}\delta m\right)^2 + \left(\dfrac{\delta b}{m}\right)^2 + \left(\dfrac{\delta \rho}{m}\right)^2}
		\end{equation}
		\begin{figure}[h]
			\centering
			\includegraphics[width=\linewidth]{../Data/Concentration/C_HCl_initial.png}
			\caption{Determinaci\'on de la concentraci\'on de una soluci\'on de HCl usando valores de densidad reportados en la literatura \cite{perry2007perry}.}
			\label{fig: HCl_density}
		\end{figure}
		
		Para obtener la concentraci\'on $0.84 \pm 0.04$ M, se usa la siguiente ecuaci\'on, la cual relaciona la concentraci\'on en fracci\'on de masa con la molaridad $[M]$:
		\begin{equation}
			[M] = 10\dfrac{[w_t]\rho}{m_m} \qquad \text{$m_m$ la masa molecular del HCl}
		\end{equation}
		
		Se tiene entonces que la incertidumbre en la concentraci\'on estar\'a dada por:
		\begin{equation}
			\delta [M] = [M]\sqrt{\delta[w_t]^2 + \delta\rho^2} 
		\end{equation}
		
		Una al\'icuota de 0,0297 g de esta soluci\'on fue disuelta en 99,5553 g de agua, dando lugar a una soluci\'on 0,2469 mM. Donde la concentraci\'on final se calcula a partir de las densidades de las soluciones inicial ($\rho_s$) y final ($\rho_f$), las masas de agua ($m_\text{\ce{H2O}}$) y de soluci\'on inicial ($m_s$) usando la siguiente ecuaci\'on:
		\begin{equation}\label{eq: concentracion_f}
			[M]_f = [M]_i\dfrac{V_s}{V_f} = [M_i]\dfrac{m_s/\rho_s}{(m_s + m_{\text{\ce{H2O}}})/\rho_f} =  [M]_i\left(\dfrac{m_s}{m_s + m_{\text{\ce{H2O}}}}\right)\left(\dfrac{\rho_s}{\rho_f}\right)
		\end{equation}
		
		Las densidades de las soluciones inicial y final se muestran en la \autoref{tb: concentracion_f}.
		
	\subsection{Soluci\'on de \ce{KHCO3} 0.17 mM}
	En un bal\'on aforado de 10 mL fueron adicionados 0,10143 g de \ce{KHCO3}, junto con 9,93551 g de \ce{H2O}. La densidad fue medida a 25 $^\circ$C y su valor fue 1,003662 g/cm$^3$. La concentraci\'on de esta soluci\'on se calcul\'o usando la siguiente ecuaci\'on:
	\begin{equation}
		[M] = \dfrac{n}{V} = \dfrac{m\rho}{m_m(m + m_{\ce{H2O}})}
	\end{equation}

	Obteniendo un valor de $0.10131\pm 0.00001$ M, donde la incertidumbre se calcula usando:
	\small
	\begin{equation}
		\delta[M] = \sqrt{\frac{m^{2}}{m_m^{2} \left(m + m_{\text{\ce{H2O}}}\right)^{2}}\delta\rho^{2} + \frac{\rho^{2} m_{\text{\ce{H2O}}}^{2}}{m_m^{2} \left(m + m_{\text{\ce{H2O}}}\right)^{4}}\delta m^{2} + \frac{\rho^{2} m^{2}}{m_m^{2} \left(m + m_{\text{\ce{H2O}}}\right)^{4}}\delta{m_{\text{\ce{H2O}}}}^{2}}
	\end{equation}
	\normalsize
	
	Posteriormente se tom\'o una al\'icuota de 0,1709 g, la cual fue dilu\'ida en 99,5657 g de agua. Usando la \autoref{eq: concentracion_f}, se obtiene una concentraci\'on de 0,17265 mM. El resumen de las cantidades usadas para la diluci\'on de las soluciones de \'acido y bicarbonato, as\'i como las densidades obtenidas a 20 $^\circ$C se muestran en la \autoref{tb: concentracion_f}.
	\begin{table}[h]
		\centering
		\caption{Densidades y masas medidas para alcanzar las soluciones con concentraciones 0,25 mM y 0,17 mM para el HCl y \ce{KHCO3}.}
		\begin{tabular}{c|cccccc}
			& $\mathbf{[M]_i}$ (M) & $\mathbf{m_s}$ (g) & $\mathbf{m_{\text{H2O}}}$ (g) & $\bm{\rho_s}$ (g/cm$^3$)& $\bm{\rho_f}$ (g/cm$^3$) & $\mathbf{[M]_f}$ (mM) \\
			\hline
			\textbf{\ce{HCl}} & $0.84 \pm 0.04$ & $0.0297$ & $99.5553$ & $1.012832$ & $0.998205$ & $0.25$ \\
			\textbf{\ce{KHCO3}} & $0.10131\pm 0.00001$ & $0.1709$ & $99.5657$ & $1.003662$ & $0.998215$ & $0.17265$ \\
			\hline
		\end{tabular}
		\label{tb: concentracion_f}
	\end{table}
	
\section{Sistema de inyecci\'on}
	El sistema de inyecci\'on consiste en un motor de pasos acoplado a un tornillo de precisión, el cual controla el desplazamiento del émbolo de la jeringa de inyección. El fluido saliente de la jeringa se desvía usando una manguera de cromatografía líquida de acero inoxidable y diámetro XXXX mm, el cual se conecta en el otro extremo a un canal que lleva el fluido hasta la celda de medición.
	
	Tanto la aguja de la jeringa como la manguera de cromatografía, corresponden con una alternativa para ingresar el fluido hasta la celda que difiere del mecanismo original en los volúmenes usados y el tipo de jeringas. Originalmente, para introducir una sustancia en la celda se hace uso de una jeringa de vidrio Hammilton de 250 $\mu$L la cual cuenta con una canula soldada en la punta de esta. Al momento de realizar el experimento, no se contaba con una jeringa de este tipo con la canula conectada, y a pesar que varios intentos fueron realizados para soldar la punta con la canula de oro y acero inoxidable no fue posible juntar las partes, en parte dado que el diámetro de la canula es considerablemente pequeño, siendo difícil de ver a simple vista. Lo anterior tiene consideraciones especiales, pues el volumen interno de la manguera cromatogr\'afica es mucho mayor, por lo cual se debe cambiar la jeringa usada, además de tener especial cuidado por los volúmenes introducidos, pues no se debe exceder la capacidad de 4 mL de la celda.
	
	\subsection{Control por software}
	
	
	\subsection{Calibraci\'on de la jeringa}
		\begin{figure}[h]
			\centering
			\includegraphics[width=\linewidth]{../Data/Syringe/syringe_cal.png}
			\caption{Curva de calibraci\'on de la jeringa usada.}
			\label{fig: calibracion_jeringa}
		\end{figure}
	
		Con el termómetro de mercurio usado en la calibración térmica de los sensores de temperatura se midió la temperatura del agua tipo 1 usada en la calibración de la jeringa, dando un valor de 17.7 $^\circ$C. El densimetro usado en la \autoref{sec: soluciones} fue configurado para realizar una lectura a esta misma temperatura, con lo cual se obtuvo un valor de densidad de XXXX g/cm$^3$.
	
\section{Realizaci\'on del experimento}

\section{Resultados}


		