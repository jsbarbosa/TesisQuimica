% Chapter 1

%\chapter{Chapter Title Here} % Main chapter title
%
%\label{Chapter1} % For referencing the chapter elsewhere, use \ref{Chapter1} 

%----------------------------------------------------------------------------------------

% Define some commands to keep the formatting separated from the content 
\newcommand{\keyword}[1]{\textit{#1}}
%\newcommand{\tabhead}[1]{\textbf{#1}}
%\newcommand{\code}[1]{\texttt{#1}}
%\newcommand{\file}[1]{\texttt{\bfseries#1}}
%\newcommand{\option}[1]{\texttt{\itshape#1}}

%----------------------------------------------------------------------------------------

\section{Introducción}
	La calorimetría es una de las áreas más viejas de la fisicoquímica. Se podría considerar que la historia de esta comienza en junio de 1783, con la presentación de \textit{Memoria del calor} (Mémoire de la Chaleur) por Lavoisier y Laplace a la Academia Francesa \cite{zielenkiewicz2006theory}. La calorimetría es el estudio de la transferencia de energía en forma de calor. Empíricamente se ha demostrado que esencialmente todos los procesos físicos y químicos están acompa\~nados de absorción o liberación de energía en forma de calor. Lo anterior hace de la calorimetría una técnica con un amplio rango de aplicaciones \cite{wadso2001standards}.

	Dentro de la calorimetría, dependiendo del rango de las potencias medidas, se tienen dos términos comunmente usados: \keyword{microcalorimetría} para el caso de experimentos realizados en el rango de los microvatios \cite{wadso2001standards, wadso2003new}, mientras que para escalas de nanovatios son usados \keyword{nanocalorímetros} \cite{wadso2003new}. El término isotérmico hace refencia a que la temperatura del calorímetro se mantiene constante o con peque\~nas fluctuaciones dentro de tolerancias estrictas \cite{wadso2001standards}.
	
	La temperatura a la que un evento en particular ocurre, o el rango en el que una reacción pasa, son características de la naturaleza e historia de la misma \cite{gaisford2016principles}.
	
	
	
	


	
	

	
\section{Justificación del proyecto}
	Understanding the interactions involving macromolecules in aqueous solution o†ers an interesting and important challenge in view of the importance of these interactions industrially, commercially and biologically. In principle calorimetric techniques o†er sensitive procedures for probing these interactions \cite{blandamer1998titration}.

	The study of the thermodynamics of physically adsorbing systems has a long and venerable history. Countless papers have been written on the subject, and conferences held and books published. In part, this is because of the universality of the phenomenon of physical adsorption on surfaces and, in part, because of a need for understanding as a basis for use in numerous industrial processes such as, for example, in the separation of gas mixtures. Wartime needs for efficient adsorbents for gas masks led to much research on the adsorptive properties of porous carbons or charcoals \cite{morrison1987calorimetry}.
	
	It is not the intention of this contribution to give an exhaustive historical survey of	calorimeters or of the application of calorimetry in the study of physical adsorption.	While some developments from the more distant past will be mentioned, much more attention
	will be given to the role that calorimetry and thermodynamics can play in helping to investigate relatively new fundamental deductions about systems with fewer than three dimensions \cite{morrison1987calorimetry}.
	
	Adiabatic calorimetry at low temperature is the classical method for heat capacity measurement, from which many other thermodynamic properties such as enthalpy, entropy, and Gibbs free energy can be calculated. These data are of significance in theoretical study, application development, and industrial production of a compound. In the present
	paper, a small-sample automated adiabatic calorimeter is
	described in more detail on the basis of our previous
	work. 1,2 At the same time, the heat capacities of the
	standard reference material R-Al 2 O 3 were measured to
	demonstrate the accuracy of this calorimeter \cite{wang2005determination}.
	
	The method for the detection of exchanged heat is
	calorimetry. Calorimetry has the advantages of being un-
	specific, non-invasive and insensitive to the electrochemical	and optical properties of the investigated system. Regarding these features calorimetry appears well suited for the online
	monitoring of bioprocesses [1] studying the growth process	of microbial cultures or for the detection of biological key components [2,3] \cite{winkelmann2004application}.
\section{Objetivos}

\section{Metodología}

\section{Cronograma}
\begin{table}[h]
	\centering
	\caption{Cronograma de actividades}
	\label{tb: cronograma}
	\begin{tabular}{|c|c|c|c|c|c|c|c|c|c|c|c|c|c|c|c|c|}
		\hline
		\rowcolor[HTML]{C0C0C0} 
		\cellcolor[HTML]{C0C0C0}                                       & \multicolumn{16}{c|}{\cellcolor[HTML]{C0C0C0}\textbf{Semana}} \\ \cline{2-17} 
		\rowcolor[HTML]{EFEFEF} 
		\multirow{-2}{*}{\cellcolor[HTML]{C0C0C0}\textbf{Actividades}} & \textbf{1} & \textbf{2} & \textbf{3} & \textbf{4} & \textbf{5} & \textbf{6} & \textbf{7} & \textbf{8} & \textbf{9} & \textbf{10} & \textbf{11} & \textbf{12} & \textbf{13} & \textbf{14} & \textbf{15} & \textbf{16} \\ \hline
		\cellcolor[HTML]{EFEFEF}
		\textbf{Revisión bibliográfica} & x & x & x & x & x & x & x & x & x & x & x & x & x & x & x & x \\ \hline
		\cellcolor[HTML]{EFEFEF}\textbf{Puesta en marcha del equipo} & x & x & x & x & x & x & & & & & & & & & & \\ \hline
		\cellcolor[HTML]{EFEFEF}\textbf{Calibración del equipo} & & & & & & & x & x & x & x & & & & & & \\ \hline
		%\cellcolor[HTML]{EFEFEF}\textbf{Toma de medidas} & & & & & & & & & & & x & x & x & & & \\ \hline
		\cellcolor[HTML]{EFEFEF}\textbf{Análisis de datos} & & & & & & & & & & & x & x & x & & & \\ \hline
		\cellcolor[HTML]{EFEFEF}\textbf{Elaboración del documento} & & & & & & x & x & x & & & & & & x & x & x \\ \hline
		\cellcolor[HTML]{EFEFEF}\textbf{Presentación del proyecto} & & & & & & & & & & & & & & & & x \\ \hline
	\end{tabular}
\end{table}
